\documentclass[11pt,nochoiceboxes,solutions,internaltesting]{exam3}

\usepackage{graphicx,xspace}
\usepackage{xcolor,multicol,wrapfig}
\setlength{\columnsep}{12pt}
\setlength{\columnseprule}{1pt}

\instructorname{Prof. Ethan L.~Miller}
\instructoremail{elm}
\emaildomain{xyz.edu}

% \usepackage{draftwatermark}
% \SetWatermarkText{For use in CMPE 110 Winter 2018 only}
% \SetWatermarkScale{0.30}
% \SetWatermarkAngle{60}

\usepackage{watermark,transparent}

\begin{comment}
\ifsolutionsonly{\watermark{
  \put(30,-380){\transparent{0.15}\rotatebox[origin=c]{60}{
    \begin{minipage}{10in}
      \centering
      \fontsize{42}{42}\selectfont
      \textsf{For use in CMPE 110 Winter 2018 only \\
      Redistribution explicitly forbidden}
    \end{minipage}
  }}
  \put(-25,-350){\fbox{\rotatebox[origin=c]{90}{\fontsize{12}{14}\selectfont\textsf{\textbf{For use in CMPS 110 Winter 2018 ONLY.  Any other use constitutes academic dishonesty and copyright violation.}}}}}
  \put(510,-350){\fbox{\rotatebox[origin=c]{270}{\fontsize{12}{14}\selectfont\textsf{\textbf{For use in CMPS 110 Winter 2018 ONLY.  Any other use constitutes academic dishonesty and copyright violation.}}}}}
}}
\end{comment}


\newenvironment{itemtight}{
  \begin{list}
    {$\bullet$}
    {\setlength{\parsep}{0ex}\setlength{\itemsep}{0ex}
      \setlength{\labelwidth}{0.30in}\setlength{\labelsep}{0.10in}
      \setlength{\leftmargin}{0.30in}\setlength{\rightmargin}{0.25in}
    }
}{
  \end{list}
}

\typein[\myexam]{Exam version (1,2,3,4): }
\setexamversion{\myexam}
%\setexamversion{3}

\copyrightname{Ethan L. Miller}

\exampagename{Page}

\begin{document}

\examversionnames{VERSION A,VERSION B,VERSION C,VERSION D}
\title{CSE 100 Final Exam---\fbox{\textbf{\examversionname}}}
\date{December 10, 2018}
\maketitle

\thispagestyle{fancy}
\setlength{\partopsep}{0pt}

\section*{Multiple Choice I}

% The correct answer must have a * in front of it.  It's acceptable to have more than one *
% on a single question.
\begin{problem}{2}
  How much wood would a woodchuck chuck if a woodchuck could chuck wood?
  \begin{multichoice}
    \item *About $10^2$ cords
    \item About $10^3$ cords
    \item About $10^4$ cords
    \item About $10^5$ cords
    \item None of the above
  \end{multichoice}
\end{problem}

%
% Specifying order=.... results in the choices being in the same order on all versions.
%
\begin{problem}{2}
  What is the airspeed velocity of an unladen swallow?
  \begin{multichoice}[order=badce]
    \item 50\,kph
    \item 70\,kph
    \item 20\,kph
    \item 30\,kph
    \item *What do you mean, an African or European swallow?
  \end{multichoice}
\end{problem}

%
% Specifying set= picks a particular set of orderings.  Ordering E ensures that choice E
% is always last, and ordering D ensures that choices D and E are always last, in that
% order.
%
\begin{problem}{2}
  What is the meaning of life, the universe, and everything?
  \begin{multichoice}[set=E]
    \item Infinity
    \item *42
    \item Heaven
    \item A towel
    \item None of the above
  \end{multichoice}
\end{problem}

%
% One can also specify all of the orderings explicitly for each version.
%
\begin{problem}{2}
  How much wood would a woodchuck chuck if a woodchuck could chuck wood?
  \begin{multichoice}[v1=abcde,v2=edcba,v3=abedc,v4=cdeba]
    \item *A
    \item B
    \item C
    \item D
  \end{multichoice}
\end{problem}

% The command \examshowanswers produces a comma-separated list of multiple choice answers,
% in the order in which they were encountered in the file.  This list ignores problem
% numbers.  Non-multiple choice problems are skipped in the list; there are no "empty"
% slots for them.
\typeout{====> Answers: \examshowanswers}

\showpoints

\end{document}
